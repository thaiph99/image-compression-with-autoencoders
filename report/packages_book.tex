%!TEX root = book_ML.tex
% \usepackage[margin=1in]{geometry}
\usepackage[T5]{fontenc}
% \setcounter{secnumdepth}{3}
\setcounter{tocdepth}{1} %% depth level of table of content with 0 - chapter only 
\usepackage[utf8]{inputenc}
\usepackage{amsmath}
\usepackage{amssymb}
% \usepackage{amssymb,amsbsy}
% \setlength{\parindent}{0em}
\setlength{\parskip}{1em}
% choose options for [] as required from the list
% in the Reference Guide, Sect. 2.2

% ******************************************************************************
%% force table caption to top 
%https://tex.stackexchange.com/questions/22751/how-to-force-table--on-top
\usepackage{floatrow}
\floatsetup[table]{capposition=top}
\usepackage[font={sf,  }, tableposition=top]{caption}
\usepackage{subcaption}
\DeclareCaptionFormat{rule}{#1#2#3\rule{\textwidth}{.0pt}}
% \DeclareCaptionFormat{norule}{}
\DeclareCaptionLabelFormat{mylabel}{#1 #2.\hspace{1.5ex}}
\captionsetup[figure]{labelformat=mylabel, labelfont = bf, justification=justified, labelsep=none}
%%% change name Table and Figure
\captionsetup[table]{name=Bảng}
\captionsetup[figure]{name=Hình}
% \captionsetup[listing]{name=Code}

% caption on side
\usepackage{floatrow}
\usepackage{sidecap}

% ******************************************************************************


\usepackage{graphicx}        % standard LaTeX graphics tool
% when including figure files
\usepackage{multicol}        % used for the two-column index
\usepackage[bottom]{footmisc}% places footnotes at page bottom
% etc.
% see the list of further useful packages
% in the Reference Guide, Sects. 2.3, 3.1-3.3
% rule after figure
% \usepackage{caption}
% \usepackage[tableposition=top]{caption}

\usepackage{xcolor}
\def\myrule{\textcolor{gray}{\rule{\textwidth}{.1pt}}}
% \def\myrulethin{\rule{\textwidth}{.01pt}}

\usepackage{multirow}
%%%%%%%%%%
\usepackage{makeidx}         % allows index generation
\makeindex             % used for the subject index
% please use the style svind.ist with
% your makeindex program


\usepackage{silence}
\WarningFilter{latex}{Composite letter}
\WarningFilter{latex}{Package hyperref Warning: Composite letter}


\usepackage{cite,url,bm}
% \usepackage{hyperref}
% \urlstyle{rm}
% \usepackage[unicode]{hyperref}
\usepackage[unicode, pdftex,
    pdfauthor={Pham Hong Thai},
    pdftitle={Autoencoders}]{hyperref}
\renewcommand\UrlFont{\sffamily}
\hypersetup{
    colorlinks=black,
    linkcolor=black,
    filecolor=magenta,
    urlcolor=blue,
}

%%%%%%%%%%%%%%%%%%%5
% \usepackage{listings}
% \usepackage{color}
% \usepackage{textcomp}
% %New colors defined below
% \definecolor{codegreen}{rgb}{0,0.6,0}
% \definecolor{codegray}{rgb}{0.5,0.5,0.5}
% \definecolor{codepurple}{rgb}{0.58,0,0.82}
% \definecolor{backcolour}{rgb}{0.95,0.95,0.92}


% \makeatletter
% \expandafter\let\csname active@char\string?\endcsname\relax
% \expandafter\let\csname active@char\string!\endcsname\relax
% \expandafter\let\csname active@char\string:\endcsname\relax


% \initiate@active@char{?}
% \initiate@active@char{!}
% \initiate@active@char{:}
% \makeatletter



\usepackage{courier}

\usepackage{listings}
\renewcommand{\lstlistingname}{Code}
% \usepackage[T1]{fontenc}
\usepackage{xcolor}
\usepackage{textcomp}
%Code listing style named "mystyle"
% \lstdefinestyle{mystyle}{
%   backgroundcolor=\color{backcolour},   commentstyle=\color{codegreen},
%   keywordstyle=\color{magenta},
%   stringstyle=\color{codepurple},
%   basicstyle=\footnotesize,
%   frame = single,
%   breakatwhitespace=false,
%   breaklines=true,
%   captionpos=b,
%   keepspaces=true,
%   numbers=none,
%   numbersep=5pt,
%   showspaces=false,
%   showstringspaces=false,
%   showtabs=false,
%   tabsize=2,
%   basicstyle=\footnotesize\ttfamily,
%   % columns = flexible, 
%   keepspaces = true,
% }

\definecolor{codegreen}{rgb}{0,0.6,0}
\definecolor{codegray}{rgb}{0.5,0.5,0.5}
\definecolor{codepurple}{rgb}{0.58,0,0.82}

\definecolor{codegreen}{rgb}{1, 1, 1}
\definecolor{codegray}{rgb}{.2, .2, .2}
\definecolor{codepurple}{rgb}{0, 0, 0}
\definecolor{codekey}{rgb}{0, 0, 0}
\definecolor{backcolour}{rgb}{1, 1, 1}


\usepackage{courier}

\lstdefinestyle{mystyle}{
    backgroundcolor=\color{backcolour},
    commentstyle=\color{codegray} \itshape,
    keywordstyle=\color{codekey} \bfseries,
    numberstyle=\tiny\color{codekey},
    stringstyle=\color{codepurple},
    basicstyle=\ttfamily\footnotesize\color{codekey},
    % basewidth  = {.5em,0.5em},
    % basicstyle=\ttfamily,
    breakatwhitespace=false,
    breaklines=true,
    captionpos=b,
    keepspaces=true,
    numbers=none,
    numbersep=5pt,
    showspaces=false,
    showstringspaces=false,
    showtabs=false,
    tabsize=4,
    frame = single,
    framesep = 7pt,
    % columns = flexible, 
    % keepspaces = false
}

%"mystyle" code listing set
\lstset{style=mystyle}





\lstset{style=mystyle}
\lstset{basicstyle=\footnotesize\ttfamily,breaklines=true}


% Default fixed font does not support bold face
\DeclareFixedFont{\ttb}{T1}{txtt}{bx}{n}{11} % for bold
\DeclareFixedFont{\ttm}{T1}{txtt}{m}{n}{10}  % for newtcbtheoremal

% Custom colors
% \usepackage{color}
\definecolor{deepblue}{rgb}{0,0,0}
\definecolor{deepred}{rgb}{0,0,0}
\definecolor{deepgreen}{rgb}{0,0,0}


% % Python style for highlighting
\newcommand\pythonstyle{\lstset{
        language=Python,
        basicstyle=\ttm,
        otherkeywords={self},             % Add keywords here
        keywordstyle=\ttb\color{deepblue},
        emph={MyClass,__init__},          % Custom highlighting
        emphstyle=\ttb\color{deepred},    % Custom highlighting style
        stringstyle=\color{deepgreen},
        frame=tb,                         % Any extra options here
        showstringspaces=false            %
    }}


\usepackage{colortbl}

% \include{myenv}
\usepackage{tcolorbox}
\usepackage{wrapfig}
\newtcolorbox{mybox}[3][]
{
    colframe = #2!25,
    colback  = #2!10,
    coltitle = blue,
    title    = \textbf{#3},
    #1,
}



%% ========= long bar notation ==============================
\makeatletter
\newsavebox\myboxA
\newsavebox\myboxB
\newlength\mylenA
\newcommand*\lbar[2][.75]{%
    \sbox{\myboxA}{$\m@th#2$}%
    \setbox\myboxB\null% Phantom box
    \ht\myboxB=\ht\myboxA%
    \dp\myboxB=\dp\myboxA%
    \wd\myboxB=#1\wd\myboxA% Scale phantom
    \sbox\myboxB{$\m@th\overline{\copy\myboxB}$}%  Overlined phantom
    \setlength\mylenA{\the\wd\myboxA}%   calc width diff
    \addtolength\mylenA{-\the\wd\myboxB}%
    \ifdim\wd\myboxB<\wd\myboxA%
        \rlap{\hskip 0.5\mylenA\usebox\myboxB}{\usebox\myboxA}%
    \else
        \hskip -0.3\mylenA\rlap{\usebox\myboxA}{\hskip 0.3\mylenA\usebox\myboxB}%
    \fi}
\makeatother

%%%%%%%%%%%% wide hat 
\def\what{\widehat}



% ******************************************************************************
% custom environments
\tcbuselibrary{theorems}
\tcbuselibrary{skins,raster}
\newtcbtheorem[number within=chapter]{mytheo}{Định lý}{colback=white,colframe=gray,fonttitle=\bfseries}{th}

\newtcbtheorem[number within=chapter]{mydef}{Định nghĩa}{colback=white,colframe=gray,fonttitle=\bfseries}{def}

\newtcbtheorem[number within=chapter]{myalg}{Thuật toán}{colback=white,colframe=gray,fonttitle=\bfseries, fontupper = \itshape}{alg}

% \newenvironment{myalg}
% {
% \begin{tcolorbox}[colback = blue!5, colframe=gray, title = Thuật toán,fonttitle=\bfseries]
% \it
% }
% {
% \end{tcolorbox}
% }

\newenvironment{mydeff}
{
    \begin{tcolorbox}[colback=white,colframe=gray, title = Chú ý,fonttitle=\bfseries]
        \it
        }
        {
    \end{tcolorbox}
}

\newenvironment{mynote}
{
    \begin{tcolorbox}[colback = white!5, colframe=gray, title =,fonttitle=\bfseries]
        \it
        }
        {
    \end{tcolorbox}
}

\newcommand{\newnote}[2]{
    % 1: title, 2: content 
    \begin{tcolorbox}[colback = white, leftrule = .3mm, rightrule = .3mm, toprule = .3mm, bottomrule = .3mm, colframe=black, title =#1,fonttitle=\bfseries]
        \it
        #2
    \end{tcolorbox}
}

\newcommand{\myeqnbox}[1]{
    % 1: content 
    \begin{tcolorbox}[toptitle = 0mm, leftrule = .3mm, rightrule = .3mm, toprule = .3mm, bottomrule = .3mm, colback = white, colframe=black, sharp corners]
        \abovedisplayskip=-10pt
        \belowdisplayskip=-30pt
        #1
    \end{tcolorbox}
}

\renewcommand{\baselinestretch}{1.50}\normalsize

\newenvironment{myfr}
{\begin{center} \it
        \begin{tcolorbox}[colback = yellow!20, colframe=yellow!45!black]
            % \includegraphics[width = 2cm]{logo.png}
            \begin{wrapfigure}{L}{0.1\textwidth}
                \vspace{-100pt}
                \href{https://google.com}{
                    \includegraphics[width=\textwidth]{pgfs/logofundaml.pdf}}
                \vspace{-30pt}
            \end{wrapfigure}
            }
            {
        \end{tcolorbox}
    \end{center}
}
% END custom environments
% ******************************************************************************

% ******************************************************************************
% definitions 
\def\etal{\textit{et al.}}

\def\ba{\mathbf{a}}
\def\bb{\mathbf{b}}
\def\bd{\mathbf{d}}
\def\be{\mathbf{e}}
\def\bm{\mathbf{m}}
\def\bK{\mathbf{K}}
\def\bk{\mathbf{k}}
\def\bM{\mathbf{M}}
\def\bp{\mathbf{p}}
\def\bq{\mathbf{q}}
\def\bx{\mathbf{x}}
\def\by{\mathbf{y}}
\def\bz{\mathbf{z}}
\def\bu{\mathbf{u}}
\def\bv{\mathbf{v}}
\def\bw{\mathbf{w}}

\def\bbx{\bar{\mathbf{x}}}
\def\bbX{\bar{\mathbf{X}}}
\def\bbw{\bar{\mathbf{w}}}

\def\bE{\mathbf{E}}
\def\bX{\mathbf{X}}
\def\bY{\mathbf{Y}}
\def\bZ{\mathbf{Z}}
\def\bA{\mathbf{A}}
\def\bB{\mathbf{B}}
\def\bC{\mathbf{C}}
\def\bP{\mathbf{P}}
\def\bQ{\mathbf{Q}}
\def\bI{\mathbf{W}}
\def\bS{\mathbf{S}}
\def\bT{\mathbf{T}}
\def\bW{\mathbf{W}}
\def\bI{\mathbf{I}}
\def\bL{\mathbf{L}}
\def\bU{\mathbf{U}}
\def\bzero{\mathbf{0}}
\def\bone{\mathbf{1}}
\def\R{\mathbb{R}}
\def\L{\mathcal{L}}
\def\S{\mathcal{S}}


\def\bmt{\left[\begin{matrix}}
            \def\bmt{\end{matrix}\right]}

\def\diag{\text{diag}}

\def\bmt{\left[\begin{matrix}}
            \def\emt{\end{matrix}\right]}

\def\blambda{\boldsymbol{\lambda}}
\def\bxi{\boldsymbol{\xi}}
\def\bSigma{\mathbf{\Sigma}}
\def\bLambda{\boldsymbol{\Lambda}}
\def\bnu{\boldsymbol{\nu}}
\def\bmu{\boldsymbol{\mu}}

% \def\dpcm{\hfill $\square$} % Điều phải chứng minh. 
\def\dpcm{} % Điều phải chứng minh. 
\def\tcr{\textcolor{red}}
\def\tcb{\textcolor{blue}}
\def\trace{\text{trace}}
\def\rank{\text{rank}}
\def\sgn{\text{sgn}}
\def\assign{\leftarrow}
\def\imply{\Rightarrow}
\def\dom{\textbf{dom}}

\def\lg{\textit{\textbf{Lời giải}}:}
\def\vd{\textbf{Ví dụ}: }
\def\kq{{{Kết quả:}}}
% ******************************************************************************

% ******************************************************************************
% Python environment
\lstnewenvironment{python}[1][]
{
    \pythonstyle
    \lstset{#1}
}
{}
% Python for external files
\newcommand\pythonexternal[2][]{{
            \pythonstyle
            \lstinputlisting[#1]{#2}}}
% Python for inline
\newcommand\pythoninline[1]{{\color{deepred}\pythonstyle\lstinline!#1!}}

\newcommand{\bi}[1]{\textit{\textbf{{#1}}}} % bold and italic 
% ******************************************************************************

% ******************************************************************************
%%%%%%%%% Header and {Footer}
\usepackage{fancyhdr}
\pagestyle{fancy}
\fancyhf{}
% \fancyhead[RE,RO]{\nouppercase\leftmark}
% \fancyhead[RE,RO]{\thepage}
\chead{\thepage}
% \fancyhead[RE,LO]{\thepage}
% \fancyfoot[RE,LO]{}
% \fancyfoot[LE,RO]{Trang \thepage}

% \rfoot{Trang \thepage}
% \rfoot{\nouppercase\leftmark}

\fancypagestyle{plain}{%
    \fancyhf{}
    \renewcommand{\headrulewidth}{0pt}
}



\renewcommand{\headrulewidth}{.2pt}
\renewcommand{\footrulewidth}{.1pt}
% ******************************************************************************

\DeclareMathOperator*{\argmin}{argmin}
\DeclareMathOperator*{\argmax}{argmax}



%%%%%% definition

% ******************************************************************************
%%%%%% chapter
\makeatletter
\def\thickhrulefill{\leavevmode \leaders \hrule height 1.2ex \hfill \kern \z@}
\def\@makechapterhead#1{
    \vspace*{10\p@}%
    {\parindent \z@ \centering \reset@font
        \thickhrulefill\quad
        \scshape\bfseries\textit{\@chapapp{}  \thechapter}
        % \scshape\bfseries{\Large CHƯƠNG  \thechapter}
        \quad \thickhrulefill
        \par\nobreak
        \vspace*{10\p@}%
        \interlinepenalty\@M
        \hrule
        \vspace*{10\p@}%
        \Huge \bfseries #1 \par\nobreak
        \par
        \vspace*{10\p@}%
        \hrule
        \vskip 100\p@
    }}
% ******************************************************************************



% ******************************************************************************
% title 
\title{
    % {\centering \bf Machine Learning cơ bản}\\
    \\

    \\
    % {\small First Edition}
    % Lần cập nhật gần nhất:
    \vspace{-1.2cm}
}




\usepackage[compact]{titlesec}
\titlespacing{\section}{0pt}{*0}{*0}
\titlespacing{\subsection}{0pt}{*0}{*0}
\titlespacing{\subsubsection}{0pt}{*0}{*0}
% ******************************************************************************

%% for book cover 
\usepackage{incgraph,tikz}
\usetikzlibrary{patterns}
\newlength{\hatchspread}
\newlength{\hatchthickness}
\newlength{\hatchshift}
\newcommand{\hatchcolor}{}
% declaring the keys in tikz
\tikzset{hatchspread/.code={\setlength{\hatchspread}{#1}},
    hatchthickness/.code={\setlength{\hatchthickness}{#1}},
    hatchshift/.code={\setlength{\hatchshift}{#1}},% must be >= 0
    hatchcolor/.code={\renewcommand{\hatchcolor}{#1}}}
% setting the default values
\tikzset{hatchspread=6pt,
    hatchthickness=0.15pt,
    hatchshift=0pt,% must be >= 0
    hatchcolor=black}
% declaring the pattern
\pgfdeclarepatternformonly[\hatchspread,\hatchthickness,\hatchshift,\hatchcolor]% variables
{custom north west lines}% name
{\pgfqpoint{\dimexpr-2\hatchthickness}{\dimexpr-2\hatchthickness}}% lower left corner
{\pgfqpoint{\dimexpr\hatchspread+2\hatchthickness}{\dimexpr\hatchspread+2\hatchthickness}}% upper right corner
{\pgfqpoint{\dimexpr\hatchspread}{\dimexpr\hatchspread}}% tile size
{% shape description
    \pgfsetlinewidth{\hatchthickness}
    \pgfpathmoveto{\pgfqpoint{0pt}{\dimexpr\hatchspread+\hatchshift}}
    \pgfpathlineto{\pgfqpoint{\dimexpr\hatchspread+0.15pt+\hatchshift}{-0.15pt}}
    \ifdim \hatchshift > 0pt
        \pgfpathmoveto{\pgfqpoint{0pt}{\hatchshift}}
        \pgfpathlineto{\pgfqpoint{\dimexpr0.15pt+\hatchshift}{-0.15pt}}
    \fi
    \pgfsetstrokecolor{\hatchcolor}
    %    \pgfsetdash{{1pt}{1pt}}{0pt}% dashing cannot work correctly in all situation this way
    \pgfusepath{stroke}
}


\tikzstyle{shaded}=[pattern = custom north west lines]

% ******************************************************************************
% section number too close to headers in table of content
% Source: https://tex.stackexchange.com/questions/219160/toc-spacing-between-number-and-header
%
% \usepackage{titletoc}

% \titlecontents{chapter}[0em]{\vspace{.25\baselineskip}}
% {\eqparbox{ch}{\bfseries\thecontentslabel}\enspace}{}
% {\hspace{.5em}\hfill\contentspage}

% \titlecontents{section}[1.8em]{\vspace{.25\baselineskip}}
% {{\thecontentslabel}\enspace}{}
% {\hspace{.5em}\titlerule*[10pt]{$\cdot$}\contentspage}

% \titlecontents{subsection}[4.5em]{\vspace{.25\baselineskip}}
% {\eqparbox{Ss}{\thecontentslabel}\enspace}{}
% {\hspace{.5em}\titlerule*[10pt]{$\cdot$}\contentspage}

% \newcommand*\l@section{\@dottedtocline{1}{1.5em}{2.3em}}
% \newcommand*\l@subsection{\@dottedtocline{2}{3.8em}{3.2em}}
% \newcommand*\l@subsubsection{\@dottedtocline{3}{7.0em}{4.1em}}
% \newcommand*\l@paragraph{\@dottedtocline{4}{10em}{5em}}
% \newcommand*\l@subparagraph{\@dottedtocline{5}{12em}{6em}}

% ******************************************************************************

%%%%%%%%%%%%%
%% Blank page 
%%%%%%%%%%%
\usepackage{afterpage}

\newcommand\blankpage{%
    \null
    \thispagestyle{empty}%
    \addtocounter{page}{-1}%
    \newpage}


%%%%%%%%%%%%%%%%%%%%%%%%%%%%%%%%%%%%%%%%%%%%%%%%%5
% boxed around eqnarray
% source: https://tex.stackexchange.com/questions/109900/how-can-i-box-multiple-aligned-equations

% \newcommand*\widefbox[1]{\fbox{\hspace{2em}#1\hspace{2em}}}

% \usepackage{amsmath}
% \usepackage{empheq}
% \usepackage[theorems,skins]{tcolorbox}

% \newtcolorbox{mymathbox}[1][]{colback=white, sharp corners, #1}

% \usepackage{amsmath}