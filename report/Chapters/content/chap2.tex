%!TEX root = ../../book_ML.tex
\chapter{Cơ sở lý thuyết}
\label{cha: chap2}
% \index{principal component analysis}
% \index{PCA -- \textit{xem} principle component analysis}
% \index{PCA}

% \index{phân tích thành phần chính -- principle component analysis}
% \index{principle component analysis -- phân tích thành phần chính}
% \index{PCA}
\section{Tổng quan về các kĩ thuật nén mất mát thông tin}

\section{Các kĩ thuật nén codec thường được sử dụng}

\section{Bộ nén tự động (Autoencoder)}

Những tiến bộ trong việc đào tạo mạng nơ-ron đã giúp cải thiện hiệu suất
trong một số nguồn cung cấp, nhưng mạng nơ-ron vẫn chưa vượt qua các codec
hiện có trong việc nén hình ảnh mất dữ liệu. Các kết quả đầu tiên đã đạt
được gần đây bằng cách sử dụng bộ mã hóa tự động - đặc biệt là trên các hình ảnh nhỏ
- và mạng nơ ron đã đạt được kết quả tiên tiến nhất trong việc nén hình ảnh
không mất dữ liệu.

Bộ mã hóa tự động (Autoencoders) có tiềm năng giải quyết nhu cầu ngày càng
tăng về các thuật
toán nén mất mát có thể thực hiện được. Tùy thuộc vào tình huống,
các bộ mã hóa và giải mã có độ phức tạp tính toán khác nhau được yêu cầu.
Khi gửi dữ liệu từ máy chủ đến thiết bị di động, có thể mong muốn ghép nối
một bộ mã hóa mạnh mẽ với một bộ giải mã ít phức tạp hơn, nhưng các yêu cầu
sẽ bị đảo ngược khi gửi dữ liệu theo hướng khác. Lượng sức mạnh tính toán và
băng thông có sẵn cũng thay đổi theo thời gian khi có công nghệ mới.
Đối với mục đích lưu trữ, thời gian mã hóa và giải mã ít quan trọng hơn
so với các ứng dụng phát trực tuyến. Cuối cùng, các thuật toán nén hiện
tại có thể không tối ưu cho các định dạng phương tiện mới như hình ảnh
trường ánh sáng, video 360 hoặc nội dung VR

\subsection{Cấu trúc bộ mã hóa tự động}
Một bộ mã hóa tự động có 3 thành phần chính : bộ mã hóa f, 
bộ giải mã g, mô hình xác suất Q



