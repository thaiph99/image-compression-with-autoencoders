
\index{giảm chiều dữ liệu -- dimensionality reduction}
\index{dimensionality reduction -- giảm chiều dữ liệu}
\index{lựa chọn đặc trưng -- feature selection}
\index{feature selection -- lựa chọn đặc trưng}
\index{trích chọn đặc trưng -- feature extraction}
\index{feature extraction -- trích chọn đặc trưng}

\chapter*{MỞ ĐẦU}
\markboth{Mở đầu}{}
\addcontentsline{toc}{chapter}{MỞ ĐẦU}
\label{part: dimred}

\section{Lý do chọn đề tài}

Với sự phát triển không ngừng của khoa học và công nghệ, đặc biệt là các định dạng phương tiện mới
công nghệ phần cứng thay đổi, cũng như các yêu cầu và loại nội dung đa dạng tạo ra nhu cầu
về các thuật toán nén có giá trị cao hơn các phương pháp nén hiện tại (codec). Chúng em lựa chọn
đề tài này nhằm tìm kiếm, xây dựng, triển khai các phương pháp nén mới tận dụng sự mạnh mẽ
của phần cứng ngày nay.

Những tiến bộ trong việc đào tạo mạng nơ-ron đã giúp cải thiện hiệu suất
trong một số nguồn cung cấp, nhưng mạng nơ-ron vẫn chưa vượt qua các codec
hiện có trong việc nén hình ảnh mất dữ liệu. Các kết quả đầu tiên đã đạt
được gần đây bằng cách sử dụng bộ mã hóa tự động - đặc biệt là trên các hình ảnh nhỏ
- và mạng nơ ron đã đạt được kết quả tiên tiến nhất trong việc nén hình ảnh
không mất dữ liệu.

% Chúng ta sẽ xem xét các phương pháp giảm chiều dữ liệu phổ biến
% nhất: \textit{phân tích thành phần chính} ({principle component analysis}) cho bài toán giảm chiều dữ liệu
% vẫn giữ tối đa lượng thông tin, và \textit{linear discriminant analysis}
% cho bài toán giữ lại những đặc trưng quan trọng nhất cho việc phân loại. Trước
% hết, chúng ta cùng tìm hiểu một phương pháp phân tích ma trận vô
% cùng quan trọng  --  \textit{phân tích giá trị suy biến} ({singular value decomposition}).

\section{Mục đích của đề tài}

\begin{itemize}
      \item Phân tích các kĩ thuật nén cũ (codec)
      \item Xây dựng, tìm kiếm các kĩ thuật, phương pháp mới sử dụng trong nén dữ liệu
            đa phương tiện hiệu quả
      \item Tận dụng sự mạnh mẽ của phần cứng.
\end{itemize}

\section{Đối tượng và phạm vi nghiên cứu của đề tài}
\subsection{Đối tượng}
\begin{itemize}
      \item Các kĩ thuật nén thường được sử dụng (DCT, Huffman, kmean, ... )
      \item Các phương pháp nén được phát triển và thể hiện tính hiệu quả trong thời gian gần đây
      \item Bộ mã hóa tự động (Autoencoder)
      \item Các phương pháp phương pháp đánh giá hiệu năng nén
\end{itemize}
\subsection{Phạm vi nghiên cứu}
\begin{itemize}
      \item Nghiên cứu tập trung chủ yếu các kĩ thuật nén có mất mát dữ liệu.
      \item Tìm kiếm các phương pháp, kĩ thuật nén liên quan đến học máy, học sâu để
            tận dụng khả năng của phần cứng.
\end{itemize}


\input{Chapters/tableofnotation}
