%!TEX root = ../../book_ML.tex
\chapter{Nghiên cứu tổng quan}
\label{cha: chap1}
% \index{principal component analysis}
% \index{PCA -- \textit{xem} principle component analysis}
% \index{PCA}

% \index{phân tích thành phần chính -- principle component analysis}
% \index{principle component analysis -- phân tích thành phần chính}
% \index{PCA}
\section{Các phương pháp nghiên cứu}

Hiện nay nén dữ liệu được chia thành 2 loại:
\begin{itemize}
      \item Nén không mất mát thông tin
      \item Nén có mất mát thông tin
\end{itemize}

\section{Ưu nhược điểm của các phương pháp}
\subsection{Nén không mất mát thông tin: }

Các thuật toán nén không mất dữ liệu thường dựa trên giả thuyết dư thừa trong dữ liệu
và thể hiện dữ liệu chính xác hơn mà không mất các thông tin. Nén mà không làm mất dữ
liệu là khả thi vì tất cả các dữ liệu thực tế đều có dư thừa. Ví dụ một hình ảnh có thể
có các vùng màu sắc không thay đổi trong nhiều pixel. Thay vì ghi nhận từng pixel
như đỏ, đỏ, đỏ... dữ liệu có thể được ghi là 279 điểm ảnh đỏ liên tiếp. Đây là một
ví dụ về run-length encoding; ngoài ra còn có rất nhiều giải thuật khác.

Dựa theo mức áp dụng thuật toán nén người ta chia nén thành các dạng sau:
\begin{itemize}
      \item Nén tệp tin: Đây là dạng thức nén truyền thống và thuật toán nén được
            áp dụng cho từng tệp tin riêng lẻ. Tuy vậy nếu 2 tệp tin giống nhau thì vẫn
            được nén 2 lần và được ghi 2 lần. Chỉ các byte trùng lắp trong 1 file được loại
            trừ để giảm kích thước. Tùy dữ liệu nhưng thông thường khả năng giảm sau khi
            nén chỉ từ 2-3 lần.
      \item Loại trừ trùng lắp file: Đây là dạng thức nén mà thuật toán nén được
            áp dụng cho nhiều tập tin. Các file giống hệt nhau sẽ chỉ được lưu một lần.
            Ví dụ một thư điện tử có tệp tin đính kèm được gửi cho 1000 người. Chỉ có một
            bản đính kèm được lưu và vì vậy có thể giảm khá nhiều. Thông thường có thể giảm
            từ 5-10 lần so với dữ liệu gốc. - Loại trừ trùng lắp ở mức sub-file: Đây là một
            dạng thức kết hợp cả nén tệp tin và loại trừ trùng lắp
\end{itemize}


\subsection{Nén có mất mát dữ liệu:}

Nén mất dữ liệu giảm số lượng bit bằng cách xác định các thông tin không cần thiết
và loại bỏ chúng.

Chuẩn nén tín hiệu số gồm có các chuẩn sau:

\begin{itemize}
      \item Chuẩn MJPEG:
            Đây là một trong những chuẩn cổ nhất mà hiện nay vẫn sử dụng. MJPEG (Morgan JPEG). Chuẩn này hiện chỉ sử dụng trong các thiết bị DVR rẻ tiền, chất lượng thấp. Không những chất lượng hình ảnh kém, tốn tài nguyên xử lý, cần nhiều dung lượng ổ chứa, và còn hay làm lỗi đường truyền.
      \item Chuẩn MPEG2:
            Chuẩn MPEG là một chuẩn thông dụng. Đã được sử dụng rộng rãi trong hơn một thập kỉ qua. Tuy nhiên, kích thước file lớn so với những chuẩn mới xuất hiện gần đây, và có thể gây khó khăn cho việc truyền dữ liệu.
      \item Chuẩn MPEG-4:
            Mpeg-4 là chuẩn cho các ứng dụng MultiMedia. Mpeg-4 trở thành một tiêu chuẩn cho nén ảnh kỹ thuật truyền hình số, các ứng dụng về đồ hoạ và Video tương tác hai chiều (Games, Videoconferencing) và các ứng dụng Multimedia tương tác hai chiều (World Wide Web hoặc các ứng dụng nhằm phân phát dữ liệu Video như truyền hình cáp, Internet Video...). Mpeg-4 đã trở thành một tiêu chuẩn công nghệ trong quá trình sản xuất, phân phối và truy cập vào các hệ thống Video. Nó đã góp phần giải quyết vấn đề về dung lượng cho các thiết bị lưu trữ, giải quyết vấn đề về băng thông của đường truyền tín hiệu Video hoặc kết hợp cả hai vấn đề trên.
      

\end{itemize}

\subsection{Kết luận}

Vì sự đa dạng cũng như đã được phát triển nhiều của các phương pháp nén có mất mát nên chúng em
đã lựa chọn loại nén này làm chủ đề nghiên cứu chính cho đề tài. 

