%!TEX root = ../../book_ML.tex
\chapter{Nghiên cứu tổng quan}
\label{cha: chap1}
% \index{principal component analysis}
% \index{PCA -- \textit{xem} principle component analysis}
% \index{PCA}

% \index{phân tích thành phần chính -- principle component analysis}
% \index{principle component analysis -- phân tích thành phần chính}
% \index{PCA}
\section{Các phương pháp nghiên cứu}

\begin{itemize}
    \item Hiện nay có 2 phương pháp nhận diện khuôn mặt được sử dụng rộng rãi nhất là:
          \begin{itemize}
              \item Nhận dạng dựa trên các đặc trưng của các phần tử trên khuôn mặt
                    (Feature base face recognition)
              \item Nhận dạng dựa trên xét tổng thể khuôn mặt (Appearance based face
                    recognition)
          \end{itemize}

    \item Ngoài ra còn một số phương pháp về loại sử dụng mô hình về khuôn mặt:
          \begin{itemize}
              \item Nhận dạng 2D: Elastics Bunch Graph, Active Appearance Model.
              \item Nhận dạng 3D: 3D Morphale Model
          \end{itemize}
\end{itemize}

\section{Ưu nhược điểm của các phương pháp}
\subsection{Nhận dạng dựa trên các đặc trưng của các phần tử trên khuôn mặt: }

Đây là phương pháp nhận dạng khuôn mặt dựa trên viện xác định các đặc trưng hình
học của các chi tiết trên một khuôn mặt (vị trí, diện tích, hình dạng của mắt,
mũi, miệng, ...) và mối quan hệ giữa chúng (khoảng cách của hai mắt, khoảng cách
của hai lông mày, ...).

Ưu điểm của phương pháp này là nó gần với cách mà con người sử dụng để nhận biết
khuôn mặt. Hơn nữa với việc xác định đặc tính cà mối quan hệ, phương pháp này có
thể cho kết quả tốt trong các trường hợp ảnh có nhiều nhiễu như bị nghiêng, bị
xoay hoặc ánh sáng thay đổi.

Nhược điểm của phương pháp này là cài đặt thuật toán phức tạp do việc xác định
mối quan hệ giữa các đặc tính sẽ khó phân biệt. Mặt khác, với các ảnh kích thước
bé thì các đặc tính sẽ khó phân biệt.


\subsection{Nhận dạng dựa trên xét tổng thể khuôn mặt:}

Đây là phương pháp xem mỗi ảnh có kích thước RxC là một vector trong không gian
RxC chiều. Ta sẽ xây dựng một không gian mới có chiều nhỏ hơn sao chi khi biểu diễn
trong không gian có các đặc điểm chính của một khuôn mặt không bị mất đi.
Trong không gian đó, các ảnh cùng một người sẽ được tập trung lại một nhóm
gần nhau và cách xa các nhóm khác.

Ưu điểm của phương pháp này là tìm được các đặc tính tiêu biểu của đối tượng cần nhận
dạng mà không cần phải xác định các thành phần và mối quan hệ giữa các thành phần đó.
Phương pháp sử dụng thuật toán có thể thực hiện tốt với các ảnh có độ phân giải cao,
thu gọn ảnh thành một ảnh có kích thước nhỏ hơn. Có thể kết hợp các phương pháp khác
như mạng Nơ-ron, Support Vector Machine.

Nhược điểm của phương pháp này phân loại theo chiều phân bố lớn nhất của vector.
Tuy nhiên, chiều phân bố lớn nhất không phải lúc nào cũng mang lại hiệu qua
tốt nhất cho bài toán nhận dạng và đặc biệt là phương pháp này rất nhạy với nhiễu.

\subsection{Kết luận}

Vì kết quả nghiên cứu cuối cùng là ứng dụng với yêu cầu về độ chính xác cao, 
khả năng thích ứng linh hoạt, hoạt động ổn định trong môi trường thực tế và 
hoạt động với các camera với độ phân giải thấp. Tôi quyết định chọn phương pháp 
nhận dạng dựạ trên xét tổng thể khuôn mặt (Appearance based face recognition).

